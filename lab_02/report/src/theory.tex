\section*{Теоретические вопросы}
\addcontentsline{toc}{section}{Теоретические вопросы}

\subsection*{1. Базис Lisp}
\addcontentsline{toc}{subsection}{1. Базис Lisp}

Базис --- минимальный набор инструментов и структкр данных языка, который
позволяет реализовать любую поставленную задачи.

Базис языка представлен:
\begin{itemize}
    \item атомами;
    \item структурами;
    \item функциями

    \verb|atom, eq, cons, car, cdr, cond, quote, lambda, eval, label|.
\end{itemize}

\subsection*{2. Классификация функций}
\addcontentsline{toc}{subsection}{2. Классификация функций}

\begin{itemize}
    \item Чистые --- не зависят от внешних, глобальных данных, не создают
          побочных эффектов.
    \item Формы:
        \begin{itemize}
            \item могут иметь переменное количество параметров;
            \item к арнументам может применяться особая обработка.
        \end{itemize}
    \item Функционалы:
        \begin{itemize}
            \item могут принимать функцию в качестве аргумента;
            \item могут возвращать функцию.
        \end{itemize}
\end{itemize}

Классификация базисных функций:
\begin{itemize}
    \item селекторы;
    \item конструкторы;
    \item предикаты;
    \item функции сравнения.
\end{itemize}

\subsection*{3. Способы создание функций}
\addcontentsline{toc}{subsection}{3. Способы создание функций}

Функия может быть определена двумя способами. С помощью $\lambda$-выражения
\verb|(lambda (|$\lambda$\verb|-list) f)|, где $\lambda$-list --- список
формальных аргументов, а \verb|f| - тело функции, или макро-определения
\verb|(defun name |$\lambda$\verb|-выражение)|, где name --- имя определяемой
функции.

\subsection*{4. Функции car и cdr, eq, eql, equal, equal}
\addcontentsline{toc}{subsection}{4. Функции car и cdr, eq, eql, equal, equal}

\verb|car| и \verb|cdr| являются базовыми функциями доступа к данным.
\verb|car| принимает точечную пару или пустой список в качестве аргумента и
возвращает первый элемент или \verb|nil|, соответственно. \verb|cdr| принимает
точечную  пару или пустой список и возвращает список состоящий из всех
элементов, кроме первого. Если в списке меньше двух элементов, то возвращается
\verb|nil|.

Функции сравнения (принимают два аргумента, перечислены по мере усложнения
проверки):
\begin{itemize}
    \item \verb|eq| корректно сравнивает два символьных атома. Так как атомы не
          дублирутюся для данного сеанса работы, то фактически сравниваются
          соответсвующие указатели;
    \item \verb|eql| корректно сравнивает атомы и числа одинакового типа;
    \item \verb|=| корректно сравнивает только числа, причем числа могут быть
          разных типов;
    \item \verb|equal| работает идентично \verb|eql|, но в дополнение умеет
          корректно сравнивать списки;
    \item \verb|equalp| корректно сравнивает любые S-выражения. 
\end{itemize}

\subsection*{5. Назначение и отличие в работе cons и list}
\addcontentsline{toc}{subsection}{5. Назначение и отличие в работе cons и list}

\verb|list| и \verb|cons| являются функциями создания списков (\verb|cons| ---
базовая, \verb|list| --- нет). Функция \verb|cons| создает списочную  ячейку  и
устанавливает два указателя на аргументы. Функция \verb|list| принимает
переменное число аргументов и возвращает список, элементы которого – переданные
в функцию аргументы.

