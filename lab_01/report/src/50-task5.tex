\leftsection{Задание}
Написать лямбда-выражение и соответствующую функцию, представить результаты в
виде списочных ячеек.

Написать функцию \verb|(f arl ar2 ar3 ar4)|, возвращающую \newline список:
\verb|((arl ar2) (ar3 ar4))|.

\begin{lstlisting}[language=Lisp]
((lambda (arl ar2 ar3 ar4)
   (cons (cons arl (cons ar2 nil)) (cons (cons ar3 (cons ar4 nil)) nil)))
 1 2 3 4)

(defun f1-cons (arl ar2 ar3 ar4)
  (cons (cons arl (cons ar2 nil)) (cons (cons ar3 (cons ar4 nil)) nil)))

((lambda (arl ar2 ar3 ar4) (list (list arl ar2) (list ar3 ar4))) 1 2 3 4)

(defun f1-list (arl ar2 ar3 ar4)
  (list (list arl ar2) (list ar3 ar4)))
\end{lstlisting}

\begin{figure}[h]
    \centering
    \def\svgwidth{\textwidth}
    \input{task-5-1.pdf_tex}
\end{figure}

\pagebreak

Написать функцию \verb|(f arl ar2)|, возвращающую \verb|((arl) (ar2))|.

\begin{lstlisting}[language=Lisp]
((lambda (arl ar2)
  (cons (cons arl nil) (cons (cons ar2 nil) nil))) 1 2)

(defun f2-cons (arl ar2)
  (cons (cons arl nil) (cons (cons ar2 nil) nil)))

((lambda (arl ar2)
   (list (list arl) (list ar2))) 1 2)

(defun f2-list (arl ar2)
  (list (list arl) (list ar2)))
\end{lstlisting}

\begin{figure}[h]
    \centering
    \def\svgwidth{0.9\textwidth}
    \input{task-5-2.pdf_tex}
\end{figure}

Написать функцию \verb|(f arl)|, возвращающую \verb|(((arl)))|.

\begin{lstlisting}[language=Lisp]
((lambda (arl) (cons (cons (cons arl nil) nil) nil)) 1)

(defun f3-cons (arl) (cons (cons (cons arl nil) nil) nil))

((lambda (arl) (list (list (list arl)))) 1)

(defun f3-list (arl) (list (list (list arl))))
\end{lstlisting}

\begin{figure}[h]
    \centering
    \def\svgwidth{0.4\textwidth}
    \input{task-5-3.pdf_tex}
\end{figure}

