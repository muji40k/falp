\leftsection{Задание}
Напишите результат вычисления выражений и объясните как он получен:

\begin{lstlisting}[language=Lisp]
(list 'Fred 'and 'Wilma)
;;; (FRED AND WILMA)
;;; Функция list создает список из переданнаых элементов

(list 'Fred '(and Wilma))
;;; (FRED (AND WILMA))
;;; Функция list создает список из переданнаых элементов, второй элемемнт -
;;; список из 2х элементов

(cons Nil Nil)
;;; (Nil)
;;; Функция cons создает списковуй ячейку из переданных аргументов.
;;; Таким образом, создается ячейка указатель cdr которой равен nil -
;;; признак конца списка, а значение cda - nil

(cons T Nil)
;;; (T)

(cons Nil T)
;;; (NIL . T)
;;; Второй элемент - не список - создается точечные пара

(list Nil)
;;; (NIL)
;;; Список из элемента nil

(cons '(T) Nil)
;;; ((T))
;;; Список, первый элемент которого - список

(list '(one two) '(free temp))
;;; ((ONE TWO) (FREE TEMP))
;;; Список, оба элемента которого - списки

(cons 'Fred '(and Wilma))
;;; (FRED AND WILMA)
;;; Хвост списка - список, поэтому полученная ячейка - список по определению

(cons 'Fred '(Wilma))
;;; (FRED WILMA)



(list Nil Nil)
;;; (NIL NIL)
;;; Список из двух элементов nil

(list T Nil)
;;; (T NIL)

(list Nil T)
;;; (NIL T)

(cons T (list Nil))
;;; (T NIL)

(list '(T) Nil)
;;; ((T) NIL)

(cons '(one two) '(free temp))
;;; ((ONE TWO) FREE TEMP)
\end{lstlisting}

