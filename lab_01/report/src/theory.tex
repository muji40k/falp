\section*{Теоретические вопросы}
\addcontentsline{toc}{section}{Теоретические вопросы}

\subsection*{1. Элементы языка: определение, синтаксис,\\представление в памяти}
\addcontentsline{toc}{subsection}{1. Элементы языка: определение, синтаксис, представление в памяти}
Вся информация (данные и программы) в Lisp представляется в виде символьных
выражений --- S-выражений.

По определению
\begin{equation*}
    \text{S-выражение} ::= \text{<атом>} | \text{<точечная пара>}.
\end{equation*}

Элементарные значения структур данных:
\begin{itemize}
    \item Атомы:
        \begin{itemize}
            \item символы (идентификаторы) --- синтаксически – набор литер
                  (букв и цифр), начинающихся с буквы,
            \item специальные символы --- {Т, Nil} (используются для обозначения
                  логических констант),
            \item самоопределимые атомы --- натуральные числа, дробные числа
                  (например 2/3), вещественные числа, строки –
                  последовательность символов, заключенных в двойные апострофы
                  (например “abc”).
        \end{itemize}
    \item Структуры:
        \begin{align*}
            \text{Точечные пары} ::= & (\text{<атом>}.\text{<атом>}) | (\text{<атом>}.\text{<точечная пара>}) | \\
                                     & | (\text{<точечная пара>}.\text{<атом>}) | \\
                                     & | (\text{<точечная пара>}.\text{<точечная пара>}),
        \end{align*}
        \begin{align*}
            \text{Список} & ::= \text{<пустой список>} | \text{<непустой список>}, \text{где} \\
            \text{<пусой сисок>} & ::= ( ) | \text{Nil}, \\
            \text{<непустой список>} & ::= (\text{<первый элемент>} . \text{<хвост>}), \\
            \text{<первый элемент>} & ::= \text{<S-выражение>}, \\
            \text{<хвост>} & ::= \text{<список>}.
        \end{align*}
\end{itemize}

Любая непустая структура Lisp в памяти представляется списковой
ячейкой, хранящей два указателя: на голову (первый элемент) \verb|car| и
хвост \verb|cdr| --- все остальное.

\begin{figure}[h]
    \centering
    \def\svgwidth{0.8\textwidth}
    \input{theor_mem.pdf_tex}
\end{figure}

\subsection*{2. Особенности языка Lisp. Структура программы.\\Символ апостроф}
\addcontentsline{toc}{subsection}{2. Особенности языка Lisp. Структура программы. Символ апостроф}

Lisp - интерпретируемый язык символьной обработки. Вся информация (данные и
программы) представляется в виде S-выражений, это даёт возможность выдать
программу за данные и заставить её менять саму себя. По умолчанию список
считается вычислимой формой, в которой первый элемент --- название функции,
остальные --- аргументы функции.

В основе языка Lisp лежит $\lambda$-исчисление, согласно которому, любые
вычислительные выражения могут быть преобразованы в вид функций от 1-го
аргумента.

Поскольку программа и данные представлены списками, то их нужно как-то различать. 
Для этого была создана функция \verb|quote|, блокирующая вычисления, а \verb|'|
--- ее сокращенное обозначение. 

Таким образом, символ апострофа \verb|'| --- функциональная блокировка,
эквивалентен функции quote. Блокирует вычисление выражения. Таким образом,
выражение воспринимается интерпретатором как данные.

\subsection*{3. Базис языка Lisp. Ядро языка}
\addcontentsline{toc}{subsection}{3. Базис языка Lisp. Ядро языка}

Базис --- минимальный набор инструментов и структкр данных языка, который
позволяет реализовать любую поставленную задачи.

Базис языка представлен:
\begin{itemize}
    \item атомами;
    \item структурами;
    \item функциями

    \verb|atom, eq, cons, car, cdr, cond, quote, lambda, eval, label|.
\end{itemize}

