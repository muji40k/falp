\section*{Задание}

Используя хвостовую рекурсию, разработать программу, позволяющую найти

\begin{enumerate}
    \item n!,
    \item n-е число Фибоначчи.
\end{enumerate}

\begin{lstlisting}[language=Prolog]
predicates
    factorial(integer, integer).
    fibonacci(integer, integer).

    factorial_inner(integer, integer, integer).
    fibonacci_inner(integer, integer, integer, integer).


clauses
    factorial(Num, _) :- 0 > Num, !, fail.
    factorial(Num, Res) :- factorial_inner(Num, 1, Res).

    factorial_inner(Num, Acc, Acc) :- 1 = Num, !.
    factorial_inner(Num, Acc, Res) :- TmpN = Num - 1, AccIn = Acc * Num,
                                      factorial_inner(TmpN, AccIn, Res).

    fibonacci(Num, _) :- 0 > Num, !, fail.
    fibonacci(0, 0) :- !.
    fibonacci(1, 1) :- !.
    fibonacci(Num, Result) :- fibonacci_inner(Num, 1, 0, Result).

    fibonacci_inner(2, P1, P2, Result) :- !, Result = P1 + P2.
    fibonacci_inner(Depth, P1, P2, Result) :- Tmp = P1 + P2,
                                              TmpD = Depth - 1,
                                              fibonacci_inner(TmpD, Tmp,
                                                              P1, Result).
\end{lstlisting}

% \begin{landscape}
% \scriptsize
% \begin{longtable}{|c|l|l|l|}
% \caption{Вопрос: grand\_mother("Mitchell Marion", GMother)} \\
% \end{longtable}
% \end{landscape}

\begin{landscape}
\scriptsize
\begin{longtable}{|c|l|l|l|}
\caption{Вопрос: factorial(3, Res)} \\
\hline
\multicolumn{1}{|l|}{№ шага} & \begin{tabular}[c]{@{}l@{}}Состояние\\ резольвенты\end{tabular}                                              & \begin{tabular}[c]{@{}l@{}}Унифицируемые\\ термы\end{tabular}                                                                                   & \begin{tabular}[c]{@{}l@{}}Дальнейшие\\ действия\end{tabular}                                                                              \\ \hline
1                            & factorial(3, Res).                                                                                           & \begin{tabular}[c]{@{}l@{}}factorial(3, Res) = factorial(Num, \_)\\ Удача\\ \{Num = 3\}\end{tabular}                                            & Редукция                                                                                                                                   \\ \hline
2                            & \begin{tabular}[c]{@{}l@{}}0 \textgreater 3,\\ !,\\ fail.\end{tabular}                                       & \begin{tabular}[c]{@{}l@{}}0 \textgreater 3\\ Неудача\end{tabular}                                                                              & \begin{tabular}[c]{@{}l@{}}Откат.\\ Достигнут конец базы знаний.\end{tabular}                                                              \\ \hline
3                            & factorial(3, Res).                                                                                           & \begin{tabular}[c]{@{}l@{}}factorial(3, Res) = factorial(Num, Res)\\ Удача\\ \{Num = 3, Res = Res\}\end{tabular}                                & Редукция                                                                                                                                   \\ \hline
4                            & factorial\_inner(3, 1, Res).                                                                                 & \begin{tabular}[c]{@{}l@{}}factorial\_inner(3, 1, Res) = factorial(Num, \_)\\ Неудача\end{tabular}                                              & \begin{tabular}[c]{@{}l@{}}Прямой ход.\\ Переход к следующему знанию.\end{tabular}                                                         \\ \hline
5                            & factorial\_inner(3, 1, Res).                                                                                 & \begin{tabular}[c]{@{}l@{}}factorial\_inner(3, 1, Res) = factorial(Num, Res)\\ Неудача\end{tabular}                                             & \begin{tabular}[c]{@{}l@{}}Прямой ход.\\ Переход к следующему знанию.\end{tabular}                                                         \\ \hline
6                            & factorial\_inner(3, 1, Res).                                                                                 & \begin{tabular}[c]{@{}l@{}}factorial\_inner(3, 1, Res) = factorial\_inner(Num, Acc, Acc)\\ Удача\\ \{Num = 3, Acc = 1, Res = Acc\}\end{tabular} & Редукция                                                                                                                                   \\ \hline
7                            & \begin{tabular}[c]{@{}l@{}}1 = 3,\\ !.\end{tabular}                                                          & \begin{tabular}[c]{@{}l@{}}1 = 3\\ Неудача\end{tabular}                                                                                         & \begin{tabular}[c]{@{}l@{}}Откат.\\ Достигнут конец базы знаний.\end{tabular}                                                              \\ \hline
8                            & factorial\_inner(3, 1, Res).                                                                                 & \begin{tabular}[c]{@{}l@{}}factorial\_inner(3, 1, Res) = factorial\_inner(Num, Acc, Res)\\ Удача\\ \{Num = 3, Acc = 1, Res = Res\}\end{tabular} & Редукция                                                                                                                                   \\ \hline
9                            & \begin{tabular}[c]{@{}l@{}}TmpN = 3 - 1,\\ AccIn = 1 * 3,\\ factorial\_inner(TmpN, AccIn, Res).\end{tabular} & \begin{tabular}[c]{@{}l@{}}TmpN = 3 - 1\\ Удача\\ \{TmpN = 2\}\end{tabular}                                                                     & \begin{tabular}[c]{@{}l@{}}Прямой ход.\\ Переход к следующему терму резольвенты.\end{tabular}                                              \\ \hline
10                           & \begin{tabular}[c]{@{}l@{}}AccIn = 1 * 3,\\ factorial\_inner(2, AccIn, Res).\end{tabular}                    & \begin{tabular}[c]{@{}l@{}}AccIn = 1 * 3\\ Удача\\ \{AccIn = 3\}\end{tabular}                                                                   & \begin{tabular}[c]{@{}l@{}}Прямой ход.\\ Переход к следующему терму резольвенты.\end{tabular}                                              \\ \hline
11                           & factorial\_inner(2, 3, Res).                                                                                 & \begin{tabular}[c]{@{}l@{}}factorial\_inner(2, 3, Res) = factorial(Num, \_)\\ Неудача\end{tabular}                                              & \begin{tabular}[c]{@{}l@{}}Прямой ход.\\ Переход к следующему знанию.\end{tabular}                                                         \\ \hline
\dots                          & \dots                                                                                                          & \dots                                                                                                                                             & \dots                                                                                                                                        \\ \hline
13                           & factorial\_inner(2, 3, Res).                                                                                 & \begin{tabular}[c]{@{}l@{}}factorial\_inner(3, 1, Res) = factorial\_inner(Num, Acc, Acc)\\ Удача\\ \{Num = 2, Acc = 3, Res = Acc\}\end{tabular} & Редукция                                                                                                                                   \\ \hline
14                           & \begin{tabular}[c]{@{}l@{}}1 = 2,\\ !.\end{tabular}                                                          & \begin{tabular}[c]{@{}l@{}}1 = 2\\ Неудача\end{tabular}                                                                                         & \begin{tabular}[c]{@{}l@{}}Откат.\\ Достигнут конец базы знаний.\end{tabular}                                                              \\ \hline
15                           & factorial\_inner(2, 3, Res).                                                                                 & \begin{tabular}[c]{@{}l@{}}factorial\_inner(2, 3, Res) = factorial\_inner(Num, Acc, Res)\\ Удача\\ \{Num = 2, Acc = 3, Res = Res\}\end{tabular} & Редукция                                                                                                                                   \\ \hline
16                           & \begin{tabular}[c]{@{}l@{}}TmpN = 2 - 1,\\ AccIn = 3 * 2,\\ factorial\_inner(TmpN, AccIn, Res).\end{tabular} & \begin{tabular}[c]{@{}l@{}}TmpN = 2 - 1\\ Удача\\ \{TmpN = 1\}\end{tabular}                                                                     & \begin{tabular}[c]{@{}l@{}}Прямой ход.\\ Переход к следующему терму резольвенты.\end{tabular}                                              \\ \hline
17                           & \begin{tabular}[c]{@{}l@{}}AccIn = 3 * 2,\\ factorial\_inner(1, AccIn, Res).\end{tabular}                    & \begin{tabular}[c]{@{}l@{}}AccIn = 3 * 2\\ Удача\\ \{AccIn = 6\}\end{tabular}                                                                   & \begin{tabular}[c]{@{}l@{}}Прямой ход.\\ Переход к следующему терму резольвенты.\end{tabular}                                              \\ \hline
18                           & factorial\_inner(1, 6, Res).                                                                                 & \begin{tabular}[c]{@{}l@{}}factorial\_inner(1, 6, Res) = factorial(Num, \_)\\ Неудача\end{tabular}                                              & \begin{tabular}[c]{@{}l@{}}Прямой ход.\\ Переход к следующему знанию.\end{tabular}                                                         \\ \hline
\dots                          & \dots                                                                                                          & \dots                                                                                                                                             & \dots                                                                                                                                        \\ \hline
20                           & factorial\_inner(1, 6, Res).                                                                                 & \begin{tabular}[c]{@{}l@{}}factorial\_inner(1, 6, Res) = factorial\_inner(Num, Acc, Acc)\\ Удача\\ \{Num = 1, Acc = 6, Res = Acc\}\end{tabular} & Редукция                                                                                                                                   \\ \hline
14                           & \begin{tabular}[c]{@{}l@{}}1 = 1,\\ !.\end{tabular}                                                          & \begin{tabular}[c]{@{}l@{}}1 = 1\\ Удача\end{tabular}                                                                                           & \begin{tabular}[c]{@{}l@{}}Прямой ход.\\ Переход к следующему терму резольвенты.\end{tabular}                                              \\ \hline
15                           & !.                                                                                                           & \begin{tabular}[c]{@{}l@{}}!\\ Удача\end{tabular}                                                                                               & \begin{tabular}[c]{@{}l@{}}Прямой ход.\\ Переход к следующему терму резольвенты.\end{tabular}                                              \\ \hline
16                           &                                                                                                              &                                                                                                                                                 & \begin{tabular}[c]{@{}l@{}}Резольвента пуста.\\ Найдено решение.\\ Откат\end{tabular}                                                      \\ \hline
17                           & !                                                                                                            &                                                                                                                                                 & \begin{tabular}[c]{@{}l@{}}В результате отката встречен предикат отсечения.\\ Запрет использования factorial\_inner.\\ Откат.\end{tabular} \\ \hline
18                           & factorial(3, Res).                                                                                           & \begin{tabular}[c]{@{}l@{}}factorial(3, Res) = fibonacci(Num, \_)\\ Неудача\end{tabular}                                                        & \begin{tabular}[c]{@{}l@{}}Прямой ход.\\ Переход к следующему знанию.\end{tabular}                                                         \\ \hline
\dots                          & \dots                                                                                                          & \dots                                                                                                                                             & \dots                                                                                                                                        \\ \hline
\end{longtable}
\end{landscape}

\begin{landscape}
\scriptsize
\begin{longtable}{|c|l|l|l|}
\caption{Вопрос: fibonacci(3, Res)} \\
\hline
\multicolumn{1}{|l|}{№ шага} & \multicolumn{1}{l|}{\begin{tabular}[c]{@{}l@{}}Состояние\\ резольвенты\end{tabular}}                        & \multicolumn{1}{l|}{\begin{tabular}[c]{@{}l@{}}Унифицируемые\\ термы\end{tabular}}                                                                                      & \multicolumn{1}{l|}{\begin{tabular}[c]{@{}l@{}}Дальнейшие\\ действия\end{tabular}}                                                         \\ \hline
\dots                          & \dots                                                                                                         & \dots                                                                                                                                                                     & \dots                                                                                                                                        \\ \hline
1                            & fibonacci(3, Res).                                                                                          & \begin{tabular}[c]{@{}l@{}}fibonacci(3, Res) = fibonacci(Num, \_)\\ Удача\\ \{Num = 3\}\end{tabular}                                                                    & Редукция                                                                                                                                   \\ \hline
2                            & \begin{tabular}[c]{@{}l@{}}0 \textgreater 3,\\ !,\\ fail.\end{tabular}                                      & \begin{tabular}[c]{@{}l@{}}0 \textgreater 3\\ Неудача\end{tabular}                                                                                                      & \begin{tabular}[c]{@{}l@{}}Откат.\\ Достигнут конец базы знаний.\end{tabular}                                                              \\ \hline
3                            & fibonacci(3, Res).                                                                                          & \begin{tabular}[c]{@{}l@{}}fibonacci(3, Res) = fibonacci(0, 0)\\ Неудача\end{tabular}                                                                                   & \begin{tabular}[c]{@{}l@{}}Прямой ход.\\ Переход к следующему знанию.\end{tabular}                                                         \\ \hline
4                            & fibonacci(3, Res).                                                                                          & \begin{tabular}[c]{@{}l@{}}fibonacci(3, Res) = fibonacci(1, 1)\\ Неудача\end{tabular}                                                                                   & \begin{tabular}[c]{@{}l@{}}Прямой ход.\\ Переход к следующему знанию.\end{tabular}                                                         \\ \hline
5                            & fibonacci(3, Res).                                                                                          & \begin{tabular}[c]{@{}l@{}}fibonacci(3, Res) = fibonacci(Num, Result)\\ Удача\\ \{Num = 3, Res = Result\}\end{tabular}                                                  & Редукция                                                                                                                                   \\ \hline
\dots                          & \dots                                                                                                         & \dots                                                                                                                                                                     & \dots                                                                                                                                        \\ \hline
6                            & fibonacci\_inner(3, 1, 0, Res).                                                                             & \begin{tabular}[c]{@{}l@{}}fibonacci\_inner(3, 1, 0, Res) = fibonacci\_inner(2, P1, P2, Result)\\ Неудача\end{tabular}                                                  & \begin{tabular}[c]{@{}l@{}}Прямой ход.\\ Переход к следующему знанию.\end{tabular}                                                         \\ \hline
7                            & fibonacci\_inner(3, 1, 0, Res).                                                                             & \begin{tabular}[c]{@{}l@{}}fibonacci\_inner(3, 1, 0, Res) = fibonacci\_inner(Depth, P1, P2, Result)\\ Удача\\ \{Depth = 3, P1 = 1, P2 = 0, Res = Resulst\}\end{tabular} & Редукция                                                                                                                                   \\ \hline
8                            & \begin{tabular}[c]{@{}l@{}}Tmp = 1 + 0,\\ TmpD = 3 - 1,\\ fibonacci\_inner(TmpD, Tmp, 1, Res).\end{tabular} & \begin{tabular}[c]{@{}l@{}}Tmp = 1 + 0\\ Удача\\ \{Tmp = 1\}\end{tabular}                                                                                               & \begin{tabular}[c]{@{}l@{}}Прямой ход.\\ Переход к следующему терму резольвенты.\end{tabular}                                              \\ \hline
9                            & \begin{tabular}[c]{@{}l@{}}TmpD = 3 - 1,\\ fibonacci\_inner(TmpD, 1, 1, Res).\end{tabular}                  & \begin{tabular}[c]{@{}l@{}}TmpD = 3 - 1\\ Удача\\ \{TmpD = 2\}\end{tabular}                                                                                             & \begin{tabular}[c]{@{}l@{}}Прямой ход.\\ Переход к следующему терму резольвенты.\end{tabular}                                              \\ \hline
\dots                          & \dots                                                                                                         & \dots                                                                                                                                                                     & \dots                                                                                                                                        \\ \hline
10                           & fibonacci\_inner(2, 1, 1, Res).                                                                             & \begin{tabular}[c]{@{}l@{}}fibonacci\_inner(2, 1, 1, Res) = fibonacci\_inner(2, P1, P2, Result)\\ Удача\\ \{P1 = 1, P2 =1, Res = Result\}\end{tabular}                  & Редукция                                                                                                                                   \\ \hline
11                           & \begin{tabular}[c]{@{}l@{}}!,\\ Res = 1 + 1.\end{tabular}                                                   & \begin{tabular}[c]{@{}l@{}}!\\ Успех\\ \{\}\end{tabular}                                                                                                                & \begin{tabular}[c]{@{}l@{}}Прямой ход.\\ Переход к следующему терму резольвенты.\end{tabular}                                              \\ \hline
12                           & Res = 1 + 1.                                                                                                & \begin{tabular}[c]{@{}l@{}}Res = 1 + 1\\ Успех\\ \{Res = 2\}\end{tabular}                                                                                               & \begin{tabular}[c]{@{}l@{}}Прямой ход.\\ Переход к следующему терму резольвенты.\end{tabular}                                              \\ \hline
13                           &                                                                                                             &                                                                                                                                                                         & \begin{tabular}[c]{@{}l@{}}Резольвента пуста.\\ Найдено решение.\\ Откат\end{tabular}                                                      \\ \hline
14                           & Res = 1 + 1.                                                                                                &                                                                                                                                                                         & \begin{tabular}[c]{@{}l@{}}Откат.\\ Достигнут конец базы знаний.\end{tabular}                                                              \\ \hline
15                           & \begin{tabular}[c]{@{}l@{}}!,\\ Res = 1 + 1.\end{tabular}                                                   &                                                                                                                                                                         & \begin{tabular}[c]{@{}l@{}}В результате отката встречен предикат отсечения.\\ Запрет использования fibonacci\_inner.\\ Откат.\end{tabular} \\ \hline
16                           & fibonacci(3, Res).                                                                                          & \begin{tabular}[c]{@{}l@{}}fibonacci(3, Res) = fibonacci\_inner(2, P1, P2, Result)\\ Неудача\end{tabular}                                                               & \begin{tabular}[c]{@{}l@{}}Прямой ход.\\ Переход к следующему знанию.\end{tabular}                                                         \\ \hline
\dots                          & \dots                                                                                                         & \dots                                                                                                                                                                     & \dots                                                                                                                                \\ \hline
\end{longtable}
\end{landscape}

