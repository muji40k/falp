\section*{Задание}

Разработать свою программу --- <<Телефонный справочник>>. Абоненты могут иметь
несколько телефонов. Протестировать работу программы, используя разные вопросы.

\begin{itemize}
    \item <<Телефонный справочник>>: Фамилия, №тел, Адрес --- структура
          (Город, Улица, №дома, №кв),
    \item <<Автомобили>>: Фамилия\_владельца, Марка, Цвет, Стоимость, Номер.
\end{itemize}

Владелец может иметь несколько телефонов, автомобилей (Факты). В разных городах
есть однофамильцы, в одном городе – фамилия уникальна.

Используя конъюнктивное правило и простой вопрос, обеспечить возможность
поиска:
\begin{itemize}
    \item По Марке и Цвету автомобиля найти Фамилию, Город, Телефон .
\end{itemize}

\lstinputlisting[language=Prolog]{../lab_01.pro}

