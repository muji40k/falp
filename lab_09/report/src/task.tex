\section*{Задание}

\begin{enumerate}
    \item Создать базу знаний «Предки» , позволяющую наиболее эффективным способом
          (за меньшее количество шагов, что обеспечивается меньшим количеством
          предложений БЗ - правил), и используя разные варианты (примеры) простого вопроса,
          (указать: какой вопрос для какого варианта) определить:
          \begin{itemize}
              \item по имени субъекта определить всех его бабушек (предки 2-го колена),
              \item по имени субъекта определить всех его дедушек (предки 2-го колена),
              \item по имени субъекта определить всех его бабушек и дедушек (предки 2-го колена),
              \item по имени субъекта определить его бабушку по материнской линии (предки 2-го колена),
              \item по имени субъекта определить его бабушку и дедушку по материнской линии (предки 2-го колена).
          \end{itemize}
          Минимизировать количество правил и количество вариантов вопросов. Использовать
          конъюнктивные правила и простой вопрос. Для одного из вариантов ВОПРОСА задания 1
          составить таблицу, отражающую конкретный порядок работы системы.
    \item Дополнить базу знаний правилами, позволяющими найти
          \begin{enumerate}
              \item Максимум из двух чисел
                  \begin{itemize}
                    \item без использования отсечения,
                    \item с использованием отсечения;
                  \end{itemize}
              \item Максимум из трех чисел
                  \begin{itemize}
                    \item без использования отсечения,
                    \item с использованием отсечения;
                  \end{itemize}
          \end{enumerate}
\end{enumerate}

\pagebreak

\begin{lstlisting}[language=Prolog]
domains
    name = string.

predicates
    mother(name, name).
    father(name, name).

    grand_parent(name, name).
    grand_mother(name, name).
    grand_father(name, name).

    mother_n(name, integer, name).

    paternal_grand_mother(name, name).
    paternal_grand_father(name, name).
    maternal_grand_mother(name, name).
    maternal_grand_father(name, name).
    maternal_grand_parent(name, name).

    max2(integer, integer, integer).
    max3(integer, integer, integer, integer).
    max2_cut(integer, integer, integer).
    max3_cut(integer, integer, integer, integer).

clauses
    % #4
    paternal_grand_mother(Child, Grand) :- father(Child, P),
                                           mother(P, Grand).
    paternal_grand_father(Child, Grand) :- father(Child, P),
                                           father(P, Grand).
    maternal_grand_mother(Child, Grand) :- mother(Child, P),
                                           mother(P, Grand).
    maternal_grand_father(Child, Grand) :- mother(Child, P),
                                           father(P, Grand).

    % #1
    grand_mother(Child, Grand) :- maternal_grand_mother(Child, Grand).
    grand_mother(Child, Grand) :- paternal_grand_mother(Child, Grand).
    % #2
    grand_father(Child, Grand) :- maternal_grand_father(Child, Grand).
    grand_father(Child, Grand) :- paternal_grand_father(Child, Grand).
    % #3
    grand_parent(Child, GMother) :- grand_mother(Child, GMother).
    grand_parent(Child, GFather) :- grand_father(Child, GFather).

    % #5
    maternal_grand_parent(Child, GMother) :-
        maternal_grand_mother(Child, GMother).
    maternal_grand_parent(Child, GFather) :-
        maternal_grand_father(Child, GFather).

    mother("Jackson Jeffery", "Calle Lori").
    mother("Alvarez Yolanda", "Mckinney Dorothy").
    mother("Held Joseph", "Cleaves Lucinda").
    mother("Delcid Sophia", "Broadus Beverly").
    mother("Harrison Marcos", "Medlin Suzanna").
    mother("Girard Deborah", "Mcbride Irene").
    mother("Mitchell Leroy", "Bucher Florence").
    mother("Suarez Marie", "Daquilante Jill").
    mother("Tan Gerald", "Smidt Jana").
    mother("Stewart Lillian", "Morales Jennifer").
    mother("Israel Dominick", "Germain Eliza").
    mother("Smith Justine", "Gibeau Cathy").
    mother("Long John", "Vazquez Linda").
    mother("Collier Laverne", "Fair Susan").
    mother("Kingsbury Douglas", "Ibarra Imelda").
    mother("Lehman Beulah", "Longo Kathleen").
    mother("Held Abel", "Delcid Sophia").
    mother("Jackson Pauline", "Alvarez Yolanda").
    mother("Mitchell Marion", "Suarez Marie").
    mother("Harrison Lorraine", "Girard Deborah").
    mother("Tan Robert", "Stewart Lillian").
    mother("Israel Alana", "Smith Justine").
    mother("Long Ronald", "Collier Laverne").
    mother("Kingsbury Kristina", "Lehman Beulah").
    mother("Held Robert", "Jackson Pauline").
    mother("Mitchell Mary", "Harrison Lorraine").
    mother("Tan Norman", "Israel Alana").
    mother("Long Dorothy", "Kingsbury Kristina").
    mother("Held Nathaniel", "Mitchell Mary").
    mother("Tan Sabrina", "Long Dorothy").
    mother("Held Tiffany", "Tan Sabrina").

    father("Jackson Jeffery", "Jackson Fredrick").
    father("Alvarez Yolanda", "Alvarez David").
    father("Held Joseph", "Held Thomas").
    father("Delcid Sophia", "Delcid Bill").
    father("Harrison Marcos", "Harrison John").
    father("Girard Deborah", "Girard George").
    father("Mitchell Leroy", "Mitchell Oliver").
    father("Suarez Marie", "Suarez Alfred").
    father("Tan Gerald", "Tan Frank").
    father("Stewart Lillian", "Stewart George").
    father("Israel Dominick", "Israel Marc").
    father("Smith Justine", "Smith Javier").
    father("Long John", "Long Donald").
    father("Collier Laverne", "Collier David").
    father("Kingsbury Douglas", "Kingsbury Scott").
    father("Lehman Beulah", "Lehman Randy").
    father("Held Abel", "Held Joseph").
    father("Jackson Pauline", "Jackson Jeffery").
    father("Mitchell Marion", "Mitchell Leroy").
    father("Harrison Lorraine", "Harrison Marcos").
    father("Tan Robert", "Tan Gerald").
    father("Israel Alana", "Israel Dominick").
    father("Long Ronald", "Long John").
    father("Kingsbury Kristina", "Kingsbury Douglas").
    father("Held Robert", "Held Abel").
    father("Mitchell Mary", "Mitchell Marion").
    father("Tan Norman", "Tan Robert").
    father("Long Dorothy", "Long Ronald").
    father("Held Nathaniel", "Held Robert").
    father("Tan Sabrina", "Tan Norman").
    father("Held Tiffany", "Held Nathaniel").

    max2(A, B, Res) :- A >= B, Res = A.
    max2(A, B, Res) :- A < B, Res = B.

    max2_cut(A, B, A) :- A >= B, !.
    max2_cut(_, B, B).

    max3(A, B, C, Res) :- A >= B, A >= C, Res = A.
    max3(A, B, C, Res) :- B > A, B >= C, Res = B.
    max3(A, B, C, Res) :- C > A, C > B, Res = C.

    max3_cut(A, B, C, A) :- A >= B, A >= C, !.
    max3_cut(_, B, C, B) :- B >= C, !.
    max3_cut(_, _, C, C).
\end{lstlisting}

\begin{landscape}
\scriptsize
\begin{longtable}{|c|l|l|l|}
\caption{Вопрос: grand\_mother("Mitchell Marion", GMother)} \\
\hline
\multicolumn{1}{|l|}{№ шага} & \begin{tabular}[c]{@{}l@{}}Состояние\\ резольвенты\end{tabular}                             & \begin{tabular}[c]{@{}l@{}}Унифицируемые\\ термы\end{tabular}                                                                                                                                & \begin{tabular}[c]{@{}l@{}}Дальнейшие\\ действия\end{tabular}                                          \\ \hline
1                            & grand\_mother("Mitchell Marion", GMother).                                                  & \begin{tabular}[c]{@{}l@{}}grand\_mother("Mitchell Marion", GMother) = paternal\_grand\_mother(Child, Grand)\\ Неудача\end{tabular}                                                          & \begin{tabular}[c]{@{}l@{}}Прямой ход.\\ Переход к следующему зананию.\end{tabular}                    \\ \hline
\dots                          & \dots                                                                                         & \dots                                                                                                                                                                                          & \dots                                                                                                    \\ \hline
5                            & grand\_mother("Mitchell Marion", GMother).                                                  & \begin{tabular}[c]{@{}l@{}}grand\_mother("Mitchell Marion", GMother) = grand\_mother(Child, Grand)\\ Удача\\ \{Child = "Mitchell Marion", GMother = Grand\}\end{tabular}                     & \begin{tabular}[c]{@{}l@{}}Редукция.\\ Прямой ход.\end{tabular}                                        \\ \hline
6                            & maternal\_grand\_mother("Mitchell Marion", GMother).                                        & \begin{tabular}[c]{@{}l@{}}maternal\_grand\_mother("Mitchell Marion", GMother) = paternal\_grand\_mother(Child, Grand)\\ Неудача\end{tabular}                                                & \begin{tabular}[c]{@{}l@{}}Прямой ход.\\ Переход к следующему зананию.\end{tabular}                    \\ \hline
\dots                          & \dots                                                                                         & \dots                                                                                                                                                                                          & \dots                                                                                                    \\ \hline
8                            & maternal\_grand\_mother("Mitchell Marion", GMother).                                        & \begin{tabular}[c]{@{}l@{}}maternal\_grand\_mother("Mitchell Marion", GMother) = maternal\_grand\_mother(Child, Grand)\\ Удача\\ \{Child = "Mitchell Marion", GMother = Grand\}\end{tabular} & \begin{tabular}[c]{@{}l@{}}Редукция.\\ Прямой ход.\end{tabular}                                        \\ \hline
9                            & \begin{tabular}[c]{@{}l@{}}mother("Mitchell Marion", P),\\ mother(P, GMother).\end{tabular} & \begin{tabular}[c]{@{}l@{}}mother("Mitchell Marion", P) = paternal\_grand\_mother(Child, Grand)\\ Неудача\end{tabular}                                                                       & \begin{tabular}[c]{@{}l@{}}Прямой ход.\\ Переход к следующему зананию.\end{tabular}                    \\ \hline
\dots                          & \dots                                                                                         & \dots                                                                                                                                                                                          & \dots                                                                                                    \\ \hline
19                           & \begin{tabular}[c]{@{}l@{}}mother("Mitchell Marion", P),\\ mother(P, GMother).\end{tabular} & \begin{tabular}[c]{@{}l@{}}mother("Mitchell Marion", P) = mother("Jackson Jeffery", "Calle Lori")\\ Неудача\end{tabular}                                                                     & \begin{tabular}[c]{@{}l@{}}Прямой ход.\\ Переход к следующему зананию.\end{tabular}                    \\ \hline
\dots                          & \dots                                                                                         & \dots                                                                                                                                                                                          & \dots                                                                                                    \\ \hline
37                           & \begin{tabular}[c]{@{}l@{}}mother("Mitchell Marion", P),\\ mother(P, GMother).\end{tabular} & \begin{tabular}[c]{@{}l@{}}mother("Mitchell Marion", P) = mother("Mitchell Marion", "Suarez Marie").\\ Удача\\ \{P = "Suarez Marie"\}\end{tabular}                                           & \begin{tabular}[c]{@{}l@{}}Прямой ход.\\ Переход к следующему терму резольвенты.\end{tabular}          \\ \hline
38                           & mother("Suarez Marie", GMother).                                                            & \begin{tabular}[c]{@{}l@{}}mother("Suarez Marie", GMother) = paternal\_grand\_mother(Child, Grand)\\ Неудача\end{tabular}                                                                    & \begin{tabular}[c]{@{}l@{}}Прямой ход.\\ Переход к следующему зананию.\end{tabular}                    \\ \hline
\dots                          & \dots                                                                                         & \dots                                                                                                                                                                                          & \dots                                                                                                    \\ \hline
48                           & mother("Suarez Marie", GMother).                                                            & \begin{tabular}[c]{@{}l@{}}mother("Suarez Marie", GMother) = mother("Jackson Jeffery", "Calle Lori")\\ Неудача\end{tabular}                                                                  & \begin{tabular}[c]{@{}l@{}}Прямой ход.\\ Переход к следующему зананию.\end{tabular}                    \\ \hline
\dots                          & \dots                                                                                         & \dots                                                                                                                                                                                          & \dots                                                                                                    \\ \hline
55                           & mother("Suarez Marie", GMother).                                                            & \begin{tabular}[c]{@{}l@{}}mother("Suarez Marie", GMother) = mother("Suarez Marie", "Daquilante Jill")\\ Удача\\ \{GMother = "Daquilante Jill"\}\end{tabular}                                & \begin{tabular}[c]{@{}l@{}}Прямой ход.\\ Найдено решение.\\ Переход к следующему зананию.\end{tabular} \\ \hline
\dots                          & \dots                                                                                         & \dots                                                                                                                                                                                          & \dots                                                                                                    \\ \hline
120                          & mother("Suarez Marie", GMother).                                                            & \begin{tabular}[c]{@{}l@{}}mother("Suarez Marie", GMother) = max3\_cut(\_, \_, C, C)\\ Неудача\end{tabular}                                                                                  & \begin{tabular}[c]{@{}l@{}}Откат.\\ Достигнут конец базы зананий.\end{tabular}                         \\ \hline
121                          & \begin{tabular}[c]{@{}l@{}}mother("Mitchell Marion", P),\\ mother(P, GMother).\end{tabular} & \begin{tabular}[c]{@{}l@{}}mother("Mitchell Marion", P) = mother("Harrison Lorraine", "Girard Deborah")\\ Неудача\end{tabular}                                                               & \begin{tabular}[c]{@{}l@{}}Прямой ход.\\ Переход к следующему зананию.\end{tabular}                    \\ \hline
\dots                          & \dots                                                                                         & \dots                                                                                                                                                                                          & \dots                                                                                                    \\ \hline
173                          & \begin{tabular}[c]{@{}l@{}}mother("Mitchell Marion", P),\\ mother(P, GMother).\end{tabular} & \begin{tabular}[c]{@{}l@{}}mother("Suarez Marie", GMother) = max3\_cut(\_, \_, C, C)\\ Неудача\end{tabular}                                                                                  & \begin{tabular}[c]{@{}l@{}}Откат.\\ Достигнут конец базы зананий.\end{tabular}                         \\ \hline
174                          & maternal\_grand\_mother("Mitchell Marion", GMother).                                        & \begin{tabular}[c]{@{}l@{}}maternal\_grand\_mother("Mitchell Marion", GMother) = maternal\_grand\_father(Child, Grand)\\ Неудача\end{tabular}                                                & \begin{tabular}[c]{@{}l@{}}Прямой ход.\\ Переход к следующему зананию.\end{tabular}                    \\ \hline
\dots                          & \dots                                                                                         & \dots                                                                                                                                                                                          & \dots                                                                                                    \\ \hline
253                          & maternal\_grand\_mother("Mitchell Marion", GMother).                                        & \begin{tabular}[c]{@{}l@{}}maternal\_grand\_mother("Mitchell Marion", GMother) = max3\_cut(\_, \_, C, C)\\ Неудача\end{tabular}                                                              & \begin{tabular}[c]{@{}l@{}}Откат.\\ Достигнут конец базы зананий.\end{tabular}                         \\ \hline
254                          & grand\_mother("Mitchell Marion", GMother).                                                  & \begin{tabular}[c]{@{}l@{}}grand\_mother("Mitchell Marion", GMother) = grand\_mother(Child, Grand)\\ Удача\\ \{Child = "Mitchell Marion", GMother = Grand\}\end{tabular}                     & \begin{tabular}[c]{@{}l@{}}Редукция.\\ Прямой ход.\end{tabular}                                        \\ \hline
255                          & paternal\_grand\_mother("Mitchell Marion", GMother).                                        & \begin{tabular}[c]{@{}l@{}}paternal\_grand\_mother("Mitchell Marion", GMother) = paternal\_grand\_mother(Child, Grand)\\ Удача\\ \{Child = "Mitchell Marion", GMother = Grand\}\end{tabular} & \begin{tabular}[c]{@{}l@{}}Редукция.\\ Прямой ход.\end{tabular}                                        \\ \hline
256                          & \begin{tabular}[c]{@{}l@{}}father("Mitchell Marion", P),\\ mother(P, GMother).\end{tabular} & \begin{tabular}[c]{@{}l@{}}father("Mitchell Marion", P) = paternal\_grand\_mother(Child, Grand)\\ Неудача\end{tabular}                                                                       & \begin{tabular}[c]{@{}l@{}}Прямой ход.\\ Переход к следующему зананию.\end{tabular}                    \\ \hline
\dots                          & \dots                                                                                         & \dots                                                                                                                                                                                          & \dots                                                                                                    \\ \hline
323                          & \begin{tabular}[c]{@{}l@{}}father("Mitchell Marion", P),\\ mother(P, GMother).\end{tabular} & \begin{tabular}[c]{@{}l@{}}father("Mitchell Marion", P) = father("Mitchell Marion", "Mitchell Leroy")\\ Удача\\ \{P = "Mitchell Leroy"\}\end{tabular}                                        & \begin{tabular}[c]{@{}l@{}}Прямой ход.\\ Переход к следующему терму резольвенты.\end{tabular}          \\ \hline
324                          & mother("Mitchell Leroy", GMother).                                                          & \begin{tabular}[c]{@{}l@{}}mother("Mitchell Leroy", GMother) = paternal\_grand\_mother(Child, Grand)\\ Неудача\end{tabular}                                                                  & \begin{tabular}[c]{@{}l@{}}Прямой ход.\\ Переход к следующему зананию.\end{tabular}                    \\ \hline
\dots                          & \dots                                                                                         & \dots                                                                                                                                                                                          & \dots                                                                                                    \\ \hline
340                          & mother("Mitchell Leroy", GMother).                                                          & \begin{tabular}[c]{@{}l@{}}mother("Mitchell Leroy", GMother) = mother("Mitchell Leroy", "Bucher Florence")\\ Удача\\ \{GMother = "Bucher Florence"\}\end{tabular}                            & \begin{tabular}[c]{@{}l@{}}Прямой ход.\\ Найдено решение.\\ Переход к следующему зананию.\end{tabular} \\ \hline
\dots                          & \dots                                                                                         & \dots                                                                                                                                                                                          & \dots                                                                                                    \\ \hline
408                          & mother("Mitchell Leroy", GMother).                                                          & \begin{tabular}[c]{@{}l@{}}mother("Mitchell Leroy", GMother) = max3\_cut(\_, \_, C, C)\\ Неудача\end{tabular}                                                                                & \begin{tabular}[c]{@{}l@{}}Откат.\\ Достигнут конец базы зананий.\end{tabular}                         \\ \hline
409                          & \begin{tabular}[c]{@{}l@{}}father("Mitchell Marion", P),\\ mother(P, GMother).\end{tabular} & \begin{tabular}[c]{@{}l@{}}father("Mitchell Marion", P) = father("Harrison Lorraine", "Harrison Marcos")\\ Неудача\end{tabular}                                                              & \begin{tabular}[c]{@{}l@{}}Прямой ход.\\ Переход к следующему зананию.\end{tabular}                    \\ \hline
\dots                          & \dots                                                                                         & \dots                                                                                                                                                                                          & \dots                                                                                                    \\ \hline
430                          & \begin{tabular}[c]{@{}l@{}}father("Mitchell Marion", P),\\ mother(P, GMother).\end{tabular} & \begin{tabular}[c]{@{}l@{}}father("Mitchell Marion", P) = max3\_cut(\_, \_, C, C)\\ Неудача\end{tabular}                                                                                     & \begin{tabular}[c]{@{}l@{}}Откат.\\ Достигнут конец базы зананий.\end{tabular}                         \\ \hline
431                          & paternal\_grand\_mother("Mitchell Marion", GMother).                                        & \begin{tabular}[c]{@{}l@{}}paternal\_grand\_mother("Mitchell Marion", GMother) = paternal\_grand\_father(Child, Grand)\\ Неудача\end{tabular}                                                & \begin{tabular}[c]{@{}l@{}}Прямой ход.\\ Переход к следующему зананию.\end{tabular}                    \\ \hline
\dots                          & \dots                                                                                         & \dots                                                                                                                                                                                          & \dots                                                                                                    \\ \hline
518                          & paternal\_grand\_mother("Mitchell Marion", GMother).                                        & \begin{tabular}[c]{@{}l@{}}paternal\_grand\_mother("Mitchell Marion", GMother) = max3\_cut(\_, \_, C, C)\\ Неудача\end{tabular}                                                              & \begin{tabular}[c]{@{}l@{}}Откат.\\ Достигнут конец базы зананий.\end{tabular}                         \\ \hline
519                          & grand\_mother("Mitchell Marion", GMother).                                                  & \begin{tabular}[c]{@{}l@{}}grand\_mother("Mitchell Marion", GMother) = grand\_father(Child, Grand)\\ Неудача\end{tabular}                                                                    & \begin{tabular}[c]{@{}l@{}}Прямой ход.\\ Переход к следующему зананию.\end{tabular}                    \\ \hline
\dots                          & \dots                                                                                         & \dots                                                                                                                                                                                          & \dots                                                                                                    \\ \hline
600                          & grand\_mother("Mitchell Marion", GMother).                                                  & grand\_mother("Mitchell Marion", GMother) = max3\_cut(\_, \_, C, C)Неудача                                                                                                                   & \begin{tabular}[c]{@{}l@{}}Завершение работы.\\ Достигнут конец базы зананий.\end{tabular}             \\ \hline
\end{longtable}
\end{landscape}

\begin{center}
\scriptsize
\begin{longtable}{|c|l|l|l|}
\caption{Вопрос: max2(1, 2, Res)} \\
\hline
\multicolumn{1}{|l|}{№ шага} & \begin{tabular}[c]{@{}l@{}}Состояние\\ резольвенты\end{tabular}         & \begin{tabular}[c]{@{}l@{}}Унифицируемые\\ термы\end{tabular}                                                   & \begin{tabular}[c]{@{}l@{}}Дальнейшие\\ действия\end{tabular}                                    \\ \hline
1                            & max2(1, 2, Res).                                                        & \begin{tabular}[c]{@{}l@{}}max2(1, 2, Res) = paternal\_grand\_mother(Child, Grand)\\ Неудача\end{tabular}       & \begin{tabular}[c]{@{}l@{}}Прямой ход.\\ Переход к следующему знанию.\end{tabular}               \\ \hline
...                          & ...                                                                     & ...                                                                                                             & ...                                                                                              \\ \hline
75                           & max2(1, 2, Res).                                                        & \begin{tabular}[c]{@{}l@{}}max2(1, 2, Res) = max2(A, B, Res)\\ Удача\\ \{A = 1, B = 2, Res = Res\}\end{tabular} & \begin{tabular}[c]{@{}l@{}}Редукция\\ Прямой ход.\end{tabular}                                   \\ \hline
76                           & \begin{tabular}[c]{@{}l@{}}1 \textgreater{}= 2,\\ Res = 1.\end{tabular} & \begin{tabular}[c]{@{}l@{}}1 \textgreater{}= 2\\ Неудача\end{tabular}                                           & \begin{tabular}[c]{@{}l@{}}Откат.\\ Достигнут конец базы знаний.\end{tabular}                    \\ \hline
77                           & max2(1, 2, Res).                                                        & \begin{tabular}[c]{@{}l@{}}max2(1, 2, Res) = max2(A, B, Res)\\ Удача\\ \{A = 1, B = 2, Res = Res\}\end{tabular} & \begin{tabular}[c]{@{}l@{}}Редукция\\ Прямой ход.\end{tabular}                                   \\ \hline
78                           & \begin{tabular}[c]{@{}l@{}}1 \textless 2,\\ Res = 2.\end{tabular}       & \begin{tabular}[c]{@{}l@{}}1 \textless 2\\ Удача\\ \{\}\end{tabular}                                            & \begin{tabular}[c]{@{}l@{}}Прямой ход.\\ Переход к следующему терму резольвенты.\end{tabular}    \\ \hline
79                           & Res = 2                                                                 & \begin{tabular}[c]{@{}l@{}}Res = 2\\ Удача\\ \{Res = 2\}\end{tabular}                                           & \begin{tabular}[c]{@{}l@{}}Откат.\\ Найдено решение.\\ Достигнут конец базы знаний.\end{tabular} \\ \hline
80                           & \begin{tabular}[c]{@{}l@{}}1 \textless 2,\\ Res = 2.\end{tabular}       &                                                                                                                 & \begin{tabular}[c]{@{}l@{}}Откат.\\ Достигнут конец базы знаний.\end{tabular}                    \\ \hline
81                           & max2(1, 2, Res).                                                        & \begin{tabular}[c]{@{}l@{}}max2(1, 2, Res) = max2\_cut(A, B, A)\\ Неудача\end{tabular}                          & \begin{tabular}[c]{@{}l@{}}Прямой ход.\\ Переход к следующему знанию.\end{tabular}               \\ \hline
...                          & ...                                                                     & ...                                                                                                             & ...                                                                                              \\ \hline
88                           & max2(1, 2, Res).                                                        & \begin{tabular}[c]{@{}l@{}}max2(1, 2, Res) = max3\_cut(\_, \_, C, C)\\ Неудача\end{tabular}                     & \begin{tabular}[c]{@{}l@{}}Завершение работы.\\ Достигнут конец базы знаний.\end{tabular}        \\ \hline
\end{longtable}
\end{center}

\begin{center}
\scriptsize
\begin{longtable}{|c|l|l|l|}
\caption{Вопрос: max2\_cut(1, 2, Res)} \\
\hline
\multicolumn{1}{|l|}{№ шага} & \begin{tabular}[c]{@{}l@{}}Состояние\\ резольвенты\end{tabular}   & \begin{tabular}[c]{@{}l@{}}Унифицируемые\\ термы\end{tabular}                                                         & \begin{tabular}[c]{@{}l@{}}Дальнейшие\\ действия\end{tabular}                                         \\ \hline
1                            & max2\_cut(1, 2, Res).                                             & \begin{tabular}[c]{@{}l@{}}max2\_cut(1, 2, Res) = paternal\_grand\_mother(Child, Grand)\\ Неудача\end{tabular}        & \begin{tabular}[c]{@{}l@{}}Прямой ход.\\ Переход к следующему знанию.\end{tabular}                    \\ \hline
...                          & ...                                                               & ...                                                                                                                   & ...                                                                                                   \\ \hline
77                           & max2\_cut(1, 2, Res).                                             & \begin{tabular}[c]{@{}l@{}}max2\_cut(1, 2, Res) = max2\_cut(A, B, A)\\ Удача\\ \{A = 1, B = 2, Res = A\}\end{tabular} & \begin{tabular}[c]{@{}l@{}}Редукция\\ Прямой ход.\end{tabular}                                        \\ \hline
78                           & \begin{tabular}[c]{@{}l@{}}1 \textgreater{}= 2,\\ !.\end{tabular} & \begin{tabular}[c]{@{}l@{}}1 \textgreater{}= 2\\ Неудача\end{tabular}                                                 & \begin{tabular}[c]{@{}l@{}}Откат.\\ Достигнут конец базы знаний.\end{tabular}                         \\ \hline
79                           & max2\_cut(1, 2, Res).                                             & \begin{tabular}[c]{@{}l@{}}max2\_cut(1, 2, Res) = max2\_cut(\_, B, B)\\ Удача\\ \{B = 2, Res = B\}\end{tabular}       & \begin{tabular}[c]{@{}l@{}}Прямой ход.\\ Найдено решение.\\ Переход к следующему знанию.\end{tabular} \\ \hline
80                           & max2\_cut(1, 2, Res).                                             & \begin{tabular}[c]{@{}l@{}}max2\_cut(1, 2, Res) = max3(A, B, C, Res)\\ Неудача\end{tabular}                           & \begin{tabular}[c]{@{}l@{}}Прямой ход.\\ Переход к следующему знанию.\end{tabular}                    \\ \hline
...                          & ...                                                               & ...                                                                                                                   & ...                                                                                                   \\ \hline
85                           & max2\_cut(1, 2, Res).                                             & \begin{tabular}[c]{@{}l@{}}max2\_cut(1, 2, Res) = max3\_cut(\_, \_, C, C)\\ Неудача\end{tabular}                      & \begin{tabular}[c]{@{}l@{}}Завершение работы.\\ Достигнут конец базы знаний.\end{tabular}             \\ \hline
\end{longtable}
\end{center}

