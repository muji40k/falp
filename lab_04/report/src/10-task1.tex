\leftsection{Задание}
Чем принципиально отличаются функции \verb|cons|, \verb|list|, \verb|append|?

\verb|(cons object1 object2)| \\
Возвращает новую ячейку, \verb|car| которой --- \verb|object1|,
а \verb|cdr| --- \verb|object2|.

\verb|(list &rest objects)| \\
Возвращает новый список, состоящий из объектов \verb|objects|.

\verb|(append &rest prolists)| \\
Возвращает список с элементами из всех списков \verb|prolists|. Последний
аргумент, который может быть любого типа, не копируется.

Каковы результаты вычисления следующих выражений?

\begin{lstlisting}[language=Lisp]
(setf lst1 '(a b c))
(setf lst2 '(d e))

(cons lst1 lst2)
;; ((A B C) D E)

(list lst1 lst2)
;; ((A B C) (D E))

(append lst1 lst2)
;; (A B C D E)
\end{lstlisting}

