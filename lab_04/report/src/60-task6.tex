\leftsection{Задание}
Написать простой вариант игры в кости, в котором бросаются две
правильные кости. Если сумма выпавших очков равна 7 или 11 —
выигрыш, если выпало (1,1) или (6,6) --- игрок имеет право снова
бросить кости, во всех остальных случаях ход переходит ко второму
игроку, но запоминается сумма выпавших очков. Если второй игрок не
выигрывает абсолютно, то выигрывает тот игрок, у которого больше
очков. Результат игры и значения выпавших костей выводить на экран с
помощью функции \verb|print|.

\begin{lstlisting}[language=Lisp]
(defun get-random-pair () (list (+ (random 6) 1) (+ (random 6) 1)))

(defun turn ()
  (let* ((pair (get-random-pair)) (sum (+ (car pair) (cadr pair))))
    (print "Выпавшие очки:")
    (print pair)
    (cond ((or (= 1 (first pair) (second pair))
               (= 6 (first pair) (second pair)))
           (print "Перебросить? [y/any]: ")
           (if (eql 'y (progn (finish-output) (read t))) (turn) sum))
          (sum))))

(defun choose-winner (turn-1 turn-2)
  (cond ((or (= 7 turn-1) (= 11 turn-1)) 1)
        ((or (= 7 turn-2) (= 11 turn-2)) 2)
        ((> turn-1 turn-2) 1)
        ((< turn-1 turn-2) 2)
        (0)))

(defun game ()
  (print "ИГРА В КОСТИ")
  (print "------------------------")
  (let ((win (choose-winner
               (progn (print "Игрок 1") (turn))
               (progn (print "------------------------")
                      (print "Игрок 2") (turn)))))
    (print "------------------------")
    (cond ((zerop win) (print "Ничья") nil)
          ((= 1 win) (print "Победитель - игрок 1") nil)
          ((= 2 win) (print "Победитель - игрок 2") nil))))
\end{lstlisting}

