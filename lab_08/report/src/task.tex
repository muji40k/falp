\section*{Задание}

Создать базу знаний <<Собственники>>, дополнив (и минимально изменив) базу
знаний, хранящую знания:
\begin{itemize}
    \item <<Телефонный справочник>>: Фамилия, Noтел, Адрес --- структура (Город,
          Улица, Noдома, Noкв),
    \item <<Автомобили>>: Фамилия\_владельца, Марка, Цвет, Стоимость, и др.,
    \item <<Вкладчики банков>>: Фамилия, Банк, счет, сумма, др., знаниями
          о дополнительной собственности владельца. Преобразовать знания об
          автомобиле к форме знаний о собственности.
\end{itemize}
Вид собственности (кроме автомобиля):
\begin{itemize}
    \item Строение, стоимость и другие его характеристики;
    \item Участок, стоимость и другие его характеристики;
    \item Водный\_транспорт, стоимость и другие его характеристики.
\end{itemize}
Описать и использовать вариантный домен: Собственность. Владелец может иметь, но
только один объект каждого вида собственности (это касается и автомобиля), или не
иметь некоторых видов собственности.
Используя конъюнктивное правило и разные формы задания одного вопроса (пояснять
для какого задания – какой вопрос),
обеспечить возможность поиска:
\begin{enumerate}
    \item Названий всех объектов собственности заданного субъекта,
    \item Названий и стоимости всех объектов собственности заданного субъекта,
    \item Разработать правило, позволяющее найти суммарную стоимость всех
          объектов собственности заданного субъекта.
\end{enumerate}

\lstinputlisting[language=Prolog]{../lab_02.pro}

 \begin{center}
    \scriptsize
    \begin{tabular}{| l | l | l |}
        \hline
        № шага & Сравниваемые термы & Дальнейшие действия \\ \hline
        1      & \makecell[l]{owner(''Fedorova'', Cost, Ownership) \\ = phone\_book(''Karpov'', ''5-449-47-81'', address(''Petuhovo'', ''Tsentral'naja'', 5, 610))) \\ Неудача} & Переход к следующему знанию \\ \hline
        \dots & \dots & \dots \\ \hline
        11     & \makecell[l]{owner(''Fedorova'', Cost, Ownership) \\ = owner(''Karpov'', 1484657, car(''Volkswagen'', ''Yellow'', ''A884AG335'')) \\ Неудача} & Переход к следующему знанию \\ \hline
        12     & \makecell[l]{owner(''Fedorova'', Cost, Ownership) \\ = owner(''Karpov'', 7408751, building(''Petuhovo'', ''Tsentralnaja'', 50)) \\ Неудача} & Переход к следующему знанию \\ \hline
        \dots & \dots & \dots \\ \hline
        15     & \makecell[l]{owner(''Fedorova'', Cost, Ownership) \\ = owner(''Fedorova'', 2094906, car(''Peugeout'', ''Black'', ''J747JU107'')) \\ Удача,\\ подстановка \{Cost = 2094906, Ownership = car(''Peugeout'', ''Black'', ''J747JU107'')\}} & Переход к следующему знанию \\ \hline
        16     & \makecell[l]{owner(''Fedorova'', Cost, Ownership) \\ = owner(''Fedorova'', 1339153, ship(''REEF Jet'', ''Blue'')) \\ Удача, \\ подстановка \{Cost = 1339153, Ownership = ship(''REEF Jet'', ''Blue'')\}} & Переход к следующему знанию \\ \hline
        17     & \makecell[l]{owner(''Fedorova'', Cost, Ownership) \\ = owner(''Jashina'', 773926, car(''Volkswagen'', ''Green'', ''J667SB575'')) \\ Неудача} & Переход к следующему знанию \\ \hline
        \dots & \dots & \dots \\ \hline
        35     & \makecell[l]{owner(''Fedorova'', Cost, Ownership) \\ = owner(''Frolov'', 2158364, car(''Volkswagen'', ''Silver'', ''O392FC202'')) \\ Неудача} & Завершение работы, исчерпаны все знания \\ \hline
    \end{tabular}
 \end{center}

